% !TeX root = ./a4_handout.tex
\documentclass[11pt]{article}
\input{\string~/.macros}
\usepackage{hyperref}
\usepackage[a4paper, total={6in, 8in}, margin=0.7in]{geometry}
\newcommand{\bheading}[1]{\textbf{(#1)}}


\begin{document}


\section*{Part A: Shape Functions for Linear Tetrahedral Finite Elements}

\subsection*{Problem 1}
Start with the continuous definition of the potential energy for a linearly elastic object given by:
\[
    E=\frac{1} {2}\int_\Omega \epsilon\colon C \colon\epsilon d\Omega    
\]
For linear elasticity we can rewrite this using the
\[
    C=\frac{Y}{(1+\mu)\cdot(1-2\mu)}
    \begin{pmatrix} 
        1-\mu & \mu & \mu & 0 &0 &0 \\
        \mu & 1-\mu & \mu & 0 &0 &0 \\
        \mu & \mu & 1-\mu & 0 & 0 &0 \\
        0 & 0 & 0 & \frac{1}{2}(1-2\mu) &0 &0 \\
        0 & & 0 & 0 &\frac{1} {2}(1-2\mu) &0 \\
        0 & 0 & 0 & 0 &0 &\frac{1}{2}(1-2\mu) 
    \end{pmatrix}    
\]
\[
    \epsilon= 
    \begin{pmatrix}
        \epsilon_{xx} & \epsilon_{yy} & \epsilon_{zz} &\epsilon_{yz} & \epsilon_{xy} & \epsilon_{xx}
    \end{pmatrix}^T
    \quad \quad \text{and} \quad \quad
    \epsilon_{ij} =
    \frac{1}{2}\left(\frac{\partial \mathbf{u}_i}{\partial\mathbf{x}_j}+\frac{\partial \mathbf{u}_j}{\partial \mathbf{x}_i}\right)
\]
Here $\mathbf{u} \in \R^3$ is the displacement at a point $\mathbf{x}$ in space. $Y$ is the stiffness of the material (Young's modulus) and $\mu$ is the poisson's ratio which determines how incompressible a material is. Start by rewriting $E$ using the matrix $C$ and the vector definition of $\epsilon$.
\[
    E = \frac{1}{2} \int_{\Omega} \epsilon^t C \epsilon d\Omega    
\]


\subsection*{Problem 2}

Next you will discretize $E$ using linear finite element shape functions on a tetrahedron. The shape functions for a linear, tetrahedral element are the barycentric coordinates of each tetrahedron \href{https://en.wikipedia.org/wiki/Barycentric_coordinate_system}{the formula here} Let's use barycentric coordinates to represent the displacement of all points inside a tetrahedron: $\mathbf{u(\mathbf{x})} = \mathbf{u}_1\lambda_1 +\mathbf{u}_2\lambda_2+\mathbf{u}_3\lambda_3+\mathbf{u}_4\lambda_4$, where $\mathbf{u}_i$ is the displacement of the $i^{th}$ vertex and $\lambda_i$ is the $i^{th}$ barycentric coordinate. You should be able to build a discrete version of $\epsilon$ using this formula. Try to express your formula in the form $B\mathbf{u}$, where $B$ is a matrix and $\mathbf{u}$ is the $12\times 1$ vector of stacked nodal displacements: $\begin{pmatrix} u1_x & u1_y &  u1_z & \cdots & u4_x & u4_y & u4_z \end{pmatrix}^T$


\section*{Part B: Implement Linear Shape Functions}

\begin{enumerate}
    \item Open '{SRC\_DIRECTORY}/A4/include/ShapeFunctionTetLinearA4.h'.
    \item  The function \texttt{*double phi(double x) \{ ... \}} returns the value of the linear shape function
    $\lambda_{VERTEX}$, evaluated at $\mathbf{x}$. Implement this function.
    \item The function \texttt{*std::array<DataType,3> dphi(double x) \{ ... \}} returns the gradient linear shape function
    $\nabla\lambda_{VERTEX}$, evaluated at $\mathbf{x}$. Implement this function.
    \item The function \texttt{*Eigen::Matrix<DataType, 3,3> F(double x, const State \&state)} returns the $3\times3$
    deformation gradient for a tetrahedral element where $F_{ij}=\frac{\partial \mathbf{u}_i}{\partial \mathbf{X}_j}$. Use \textbf{dphi} to implement this method.
\end{enumerate}

\section*{Part C: Implement Quadrature and Simulate}

\begin{enumerate}
    \item Open '{SRC\_DIRECTORY}/A4/include/QuadratureLinearElasticity.h'
    \item Complete the \textbf{getEnergy} function to return the value of $E$ from Part A.
    \item Compute the stiffness matrix and the forces from $E$. The forces are the negative gradient of $E$ wrt to
    the degrees of freedom, $\mathbf{u}$ and the stiffness matrix is the hessian matrix, i.e $K_{ij} = - \frac{\partial^2 E}{\partial \mathbf{u}_i \partial \mathbf{u}_j}$. Derive these terms and complete the \textbf{getGradient} and \textbf{getHessian} functions.
    \item Run the assignment code which will compute the stress on a test object and plot the result.
\end{enumerate}


\end{document}

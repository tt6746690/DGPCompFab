\documentclass[11pt]{article}
\input{\string~/.macros}
\usepackage{hyperref}
\usepackage[a4paper, total={6in, 8in}, margin=0.7in]{geometry}
\newcommand{\bheading}[1]{\textbf{(#1)}}

\setcounter{MaxMatrixCols}{20}

\begin{document}


\section*{Part A: Shape Functions for Linear Tetrahedral Finite Elements}

\subsection*{Problem 1}
Start with the continuous definition of the potential energy for a linearly elastic object given by:
\[
    E=\frac{1} {2}\int_\Omega \epsilon\colon C \colon\epsilon d\Omega    
\]
For linear elasticity we can rewrite this using the
\[
    C=\frac{Y}{(1+\mu)\cdot(1-2\mu)}
    \begin{pmatrix} 
        1-\mu & \mu & \mu & 0 &0 &0 \\
        \mu & 1-\mu & \mu & 0 &0 &0 \\
        \mu & \mu & 1-\mu & 0 & 0 &0 \\ 
        0 & 0 & 0 & \frac{1}{2}(1-2\mu) &0 &0 \\
        0 & & 0 & 0 &\frac{1} {2}(1-2\mu) &0 \\
        0 & 0 & 0 & 0 &0 &\frac{1}{2}(1-2\mu) 
    \end{pmatrix}    
\]
\[
    \epsilon= 
    \begin{pmatrix}
        \epsilon_{xx} & \epsilon_{yy} & \epsilon_{zz} &\epsilon_{yz} & \epsilon_{xy} & \epsilon_{xx}
    \end{pmatrix}^T 
    \quad \quad \text{and} \quad \quad
    \epsilon_{ij} =
    \frac{1}{2}\left(\frac{\partial \mathbf{u}_i}{\partial\mathbf{x}_j}+\frac{\partial \mathbf{u}_j}{\partial \mathbf{x}_i}\right) 
\]
Here $\mathbf{u} \in \R^3$ is the displacement at a point $\mathbf{x}$ in space. $Y$ is the stiffness of the material (Young's modulus) and $\mu$ is the poisson's ratio which determines how incompressible a material is. Start by rewriting $E$ using the matrix $C$ and the vector definition of $\epsilon$.
\[
    E = \frac{1}{2} \int_{\Omega} \epsilon^t C \epsilon d\Omega    
\]
  
 
\subsection*{Problem 2}

Next you will discretize $E$ using linear finite element shape functions on a tetrahedron. The shape functions for a linear, tetrahedral element are the barycentric coordinates of each tetrahedron \href{https://en.wikipedia.org/wiki/Barycentric_coordinate_system}{the formula here} Let's use barycentric coordinates to represent the displacement of all points inside a tetrahedron: $\mathbf{u(\mathbf{x})} = \mathbf{u}_1\lambda_1 +\mathbf{u}_2\lambda_2+\mathbf{u}_3\lambda_3+\mathbf{u}_4\lambda_4$, where $\mathbf{u}_i$ is the displacement of the $i^{th}$ vertex and $\lambda_i$ is the $i^{th}$ barycentric coordinate. You should be able to build a discrete version of $\epsilon$ using this formula. Try to express your formula in the form $B\mathbf{u}$, where $B$ is a matrix and $\mathbf{u}$ is the $12\times 1$ vector of stacked nodal displacements: $\begin{pmatrix} u_{1x} & u_{1y} &  u_{1z} & \cdots & u_{4x} & u_{4y} & u_{4z} \end{pmatrix}^T$
$ $\\
Consider a linear system represent coordinate transformation from barycentric coordinate $(\lambda_1,\lambda_2,\lambda_3,\lambda_4)$ to cartesian coordinate $(x,y,z)$ for a point $\bx$ inside an element
\[
    \begin{pmatrix}
        1 \\
        x \\
        y \\
        z \\
    \end{pmatrix}    
    = 
    \begin{pmatrix}
        1 & 1 & 1 & 1 \\
        x_1 & x_2 & x_3 & x_4 \\
        y_1 & y_2 & y_3 & y_4 \\
        z_1 & z_2 & z_3 & z_4 \\
    \end{pmatrix}
    \begin{pmatrix}
        \lambda_1 \\
        \lambda_2 \\
        \lambda_3 \\
        \lambda_4
    \end{pmatrix}
\]
We can derive the inverse transformation by inverting the matrix
\[
    \begin{pmatrix}
        \lambda_1 \\
        \lambda_2 \\
        \lambda_3 \\
        \lambda_4
    \end{pmatrix}
    = 
    \frac{1}{6V}
    \begin{pmatrix}
        6V_1 & a_1 & b_1 & c_1 \\
        6V_2 & a_2 & b_2 & c_2 \\
        6V_3 & a_3 & b_3 & c_3 \\
        6V_4 & a_4 & b_4 & c_4 \\
    \end{pmatrix}
    \begin{pmatrix}
        1 \\
        x \\
        y \\
        z \\
    \end{pmatrix}    
\]
From these relationships, we can derive partial derivatives with respect to the coordinates 
\[
    \frac{\partial \lambda_i}{\partial x} = \frac{1}{6V} a_i 
    \quad \quad 
    \frac{\partial \lambda_i}{\partial y} = \frac{1}{6V} b_i
    \quad \quad
    \frac{\partial \lambda_i}{\partial z} = \frac{1}{6V} c_i 
\]
We have the following formula for computing partial derivatives of displace with respect to nodal displacement at the vertices. As an example, we can compute $\textstyle \frac{\partial \bu_x}{\partial x}$
\begin{align*}
    \frac{\partial \bu_x}{\partial x}
    &=    u_{1x} \frac{\partial \lambda_1}{\partial x} 
        + u_{2x} \frac{\partial \lambda_2}{\partial x}
        + u_{3x} \frac{\partial \lambda_3}{\partial x}
        + u_{4x} \frac{\partial \lambda_4}{\partial x} \\
    &= \frac{1}{6V} \p{
        a_1 u_{1x} + a_2 u_{2x} + a_3 u_{3x} + a_4 u_{4x}
    }    
\end{align*}
% \[
%     \epsilon
%     = 
%     \begin{pmatrix}
%         \frac{\partial \bu_x}{\partial \bx_x} \\
%         \frac{\partial \bu_y}{\partial \bx_y} \\
%         \frac{\partial \bu_z}{\partial \bx_z} \\
%         \frac{1}{2} \p{ \frac{\partial \bu_y }{\partial \bx_z} + \frac{\partial \bu_z}{\partial \bx_y} } \\
%         \frac{1}{2} \p{ \frac{\partial \bu_x }{\partial \bx_z} + \frac{\partial \bu_z}{\partial \bx_x} } \\
%         \frac{1}{2} \p{ \frac{\partial \bu_x }{\partial \bx_y} + \frac{\partial \bu_y}{\partial \bx_x} }
%     \end{pmatrix}
%     =
%     B
%     \begin{pmatrix}
%         u_{1x} \\
%         u_{1y} \\ 
%         u_{1z} \\
%         u_{2x} \\
%         u_{2y} \\ 
%         u_{2z} \\
%         u_{3x} \\
%         u_{3y} \\ 
%         u_{3z} \\
%         u_{4x} \\
%         u_{4y} \\ 
%         u_{4z} \\
%     \end{pmatrix}
% \]
We can find $B$ such that $\epsilon = B\bu$
where 
\[
    B = 
    \frac{1}{6V}
    \begin{pmatrix}
        a_1 & 0 & 0 & a_2 & 0 & 0 & a_3 & 0 & 0 & a_4 & 0 & 0 \\
        0 & b_1 & 0 & 0 & b_2 & 0 & 0 & b_3 & 0 & 0 & b_4 & 0 \\
        0 & 0 & c_1 & 0 & 0 & c_2 & 0 & 0 & c_3 & 0 & 0 & c_4 \\
        0 & \frac{1}{2} b_1 & \frac{1}{2} c_1 & 0 & \frac{1}{2} b_2 & \frac{1}{2} c_2 & 0 & \frac{1}{2} b_3 & \frac{1}{2}c_3 & 0 & \frac{1}{2} b_4 &\frac{1}{2} c_4 \\
        \frac{1}{2} a_1 & 0 & \frac{1}{2} c_1 & \frac{1}{2} a_2 & 0 & \frac{1}{2} c_2 & \frac{1}{2} a_3 & 0 & \frac{1}{2} c_3 & \frac{1}{2} a_4 & 0 & \frac{1}{2} c_4 \\
        \frac{1}{2}a_1 & \frac{1}{2}b_1 & 0 & \frac{1}{2}a_2 & \frac{1}{2}b_2 & 0 & \frac{1}{2}a_3 & \frac{1}{2}b_3 & 0 & \frac{1}{2}a_4 & \frac{1}{2}b_4 & 0 \\
    \end{pmatrix}  
\]
Therefore we can express potential energy for one element as follows
\begin{align*}
    E 
    &= \frac{1}{2} \int_{\Omega} \epsilon^t C \epsilon d\Omega \\
    &= \frac{1}{2} \int_{\Omega} \bu^t B^t C B \bu d\Omega \\
    &= \frac{1}{2}\bu VB^t C B \bu
\end{align*}
where we assume material dependent variables, i.e. $Y$ and $\mu$, are constant inside an element in last step

 

\section*{Part B: Implement Linear Shape Functions}

\begin{enumerate}
    \item Open '{SRC\_DIRECTORY}/A4/include/ShapeFunctionTetLinearA4.h'.
    \item  The function \texttt{*double phi(double x) \{ ... \}} returns the value of the linear shape function
    $\lambda_{VERTEX}$, evaluated at $\mathbf{x}$. Implement this function.
    \item The function \texttt{*std::array<DataType,3> dphi(double x) \{ ... \}} returns the gradient linear shape function
    $\nabla\lambda_{VERTEX}$, evaluated at $\mathbf{x}$. Implement this function.
    \item The function \texttt{*Eigen::Matrix<DataType, 3,3> F(double x, const State \&state)} returns the $3\times3$
    deformation gradient for a tetrahedral element where $F_{ij}=\frac{\partial \mathbf{u}_i}{\partial \mathbf{X}_j}$. Use \textbf{dphi} to implement this method.
\end{enumerate}

\section*{Part C: Implement Quadrature and Simulate}

\begin{enumerate}
    \item Open '{SRC\_DIRECTORY}/A4/include/QuadratureLinearElasticity.h'
    \item Complete the \textbf{getEnergy} function to return the value of $E$ from Part A.
    \item Compute the stiffness matrix and the forces from $E$. The forces are the negative gradient of $E$ wrt to
    the degrees of freedom, $\mathbf{u}$ and the stiffness matrix is the hessian matrix, i.e $K_{ij} = - \frac{\partial^2 E}{\partial \mathbf{u}_i \partial \mathbf{u}_j}$. Derive these terms and complete the \textbf{getGradient} and \textbf{getHessian} functions.
    \item Run the assignment code which will compute the stress on a test object and plot the result.
\end{enumerate}


\end{document}
